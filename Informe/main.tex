\documentclass{article}
\usepackage[utf8]{inputenc}
\usepackage[spanish]{babel}
\usepackage{listings}
\usepackage{graphicx}
\graphicspath{ {images/} }
\usepackage{cite}

\begin{document}

\begin{titlepage}
    \begin{center}
        \vspace*{1cm}
            
        \Huge
        \textbf{Informe}
            
        \vspace{0.5cm}
        \LARGE
        Parcial 2 de informatica.
            
        \vspace{1.5cm}
            
        \text
        
        
        \textbf{Cristian Martinez De La Ossa}
        
            
        \vfill
            
        \vspace{0.8cm}
            
        \Large
        Despartamento de Ingeniería Electrónica y Telecomunicaciones\\
        Universidad de Antioquia\\
        Medellín\\
        Septiembre de 2021
            
    \end{center}
\end{titlepage}

\tableofcontents
\newpage
\section{Introducción}\label{intro}
Resolver este parcial sera un reto bastante interesante para mi ya que es la primera vez que me enfrento con el tema de manipulacion de pixeles de una imagen, creo iniciare investigando mucho mas a fondo sobre ese tema, buscare en foros, youtube y libros, y una vez bien familiarizado con el tema empleare todo el conocimiento que he recibido a lo largo del curso para ir construyendo poco a poco el rompecabezas que necesito armar.


\section{Analisis del problema} \label{contenido}
Lo primero que estoy haciendo es analizar correctamente el problema planteado en el parcial, y segun lo que he analizado me he dado cuenta que debo traer a mi memoria todo el conocimiento de manipulacion de matrices y arreglos, tambien veo que voy a necesitar recurrir a las matematicas, quizas deba ir pensando en el uso de algebra, calculo, vectores, etc.
Asi que creo que deberia dedicarle un buen tiempo diario a la etapa de esperimentar con diferenten tecnicas para ir acercandome a la solucion mas acertada.


\section{Ideas y propuestas}
Hasta el momento estoy trabajando en el diseño del codigo mas sencillo, lo que yo llamaria el esqueleto, para luego iniciar con el nucleo o corazon del programa que seria esa parte de crear el algoritmo de escalamiento de la imagen.
Tengo pensado en crear tres vectores con llados ROJO, VERDE y AZUL, donde cada uno tendra las intencidades de color rojo,verde, y azul. luego buscar ir reduciendo en la misma proporcion el tamaño de esos vectores, auque todabia debo analizar bien como lo hare para no hechar a perder la imagen original.

\newpage
\section{Conclusión}
Pienso que este parcial sera de mucha ayuda para mi crecimiento profesional en el ambito de la programacion y para todo aquello que nos demanda poner en marcha nuestra habilidad como ingenieros.

\end{document}